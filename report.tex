\documentclass{scrartcl}
\usepackage{fullpage}
\usepackage{graphicx}
\graphicspath{ {../figures/} }
%\usepackage{multicol}
%\setlength{\columnsep}{1cm}

%\renewcommand{\baselinestretch}{2}
\author{Utkarsh Anand, A0012213J}
\title{PC5213 - Advanced Biophysics}
\subtitle{Monte Carlo Simulation of the WLC Model for DNA}
\date{\today}
\begin{document}
\maketitle

%\begin{multicols}{2}
\section{Introduction}
\label{introduction}
Monte Carlo simulations are a very useful tool to study a variety of problems which can otherwise be computationally impossible to study. This algorithm was first applied to study a statistical physics problem, and since then it has been applied to a variety of scientific problems. For instance, a DNA, which is a long polymer, can be discretized into smaller rods and the confirmation of rods can be changed. A typical DNA has a persistent length of 50 nm. Asuming that the total length of DNA is 100 nm, it can be discretized into 100 individual rods of length 1 nm each which are connected to each other at the ends. The total number of possible combinations for these rods to make any conformation will be $10^{100}$, which is a huge number the modern day computers can take decades to generate all the possible configurations. For this project, DNA of varying length were genrated using Monte Carlo simulation. The corresponding results and analysis are discussed further.

\section{Notation}
\label{notation}
\begin{tabular}{|l|l|}
\hline
\textbf{Symbol} & \textbf{Definition}\\
\hline
$B$ & Persistent length of DNA\\
$M$ & Number of beads\\
$d$ & Distance between two neighboring beads\\
$E_{\textrm{bend}}$ & Bending energy\\
$k_B$ & Boltzmann's constant\\
$T$ & Temperature\\
\hline
\end{tabular}

\section{Method}
\label{method}
All the simulation codes \ref{} are written in python using the numpy, mpi, and matplotlib libraries. The $B$ and $M$ were fixed as 50 nm and 200 respectively. $d$ took the values 0.5, 1.0, 1.5, 2.0, 2.5, 3.0, 3.5, 4.0, 4.5, and 5.0 nm. DNA simulation was done in 2D and 3D using these steps.

\begin{enumerate}
\item $M$ beads were placed on the x-axis such that they are equally spaced, the first bead is at origin, and the last bead is at $Md$.
\item Random perturbations ~\ref{figure1} were done to the DNA using either
	\begin{enumerate}
	\item \textit{Chain bending}. A bead $i$ was randomly chosen, and all the beads after it were rotated by an angle $\delta \in U(-\alpha,\alpha)$ about a randomly oriented axis passing through the center of the bead $i$. 
	\item \textit{Crank-shaft rotation}. Beads $i$ and $j$ are randomly chosen Two beads were randomly chosen and all the beads in between were  rotated by an angle $\delta \in U(-\alpha,\alpha)$ about the axis passing through the beads $i$ and $j$. Note that this type of perturbation is valid only for 3D confirmations.
	\end{enumerate}
\item After the perturbation bending energy was calculated using $$E_{\textrm{bend}} = \frac{k_B T B}{2 d} \sum_{i=2}^{M-1} \left( \hat{t}_{i,i+1} - \hat{t}_{i-1,i} \right)^2$$.
	\begin{enumerate}
	\item If $\Delta E_{\textrm{bend}} <= 0$, the perturbation is accepted.
	\item If $\Delta E_{\textrm{bend}} > 0$, then the perturbation is accepted if $r < \exp \left( -\frac{\Delta E_{\textrm{bend}}}{k_B T} \right)$, where $r \in U(0,1)$.
	\end{enumerate}
\item The DNA is subject to $10^5$ such perturbations and $\alpha$ is chosen such that the accpetance probability of perturbations is $\approx 0.5$. Figure ~\ref{figure2} shows the acceptance probability for varying $d$ values as $\alpha$ increases.
\item This simulation is done for $10^4$ DNA using the above outlined steps and the tangent correlation, bending angle ditributiion, and end-to-end distance are calcuated.
\end{enumerate}

The python code was executed on  2670 GHz processor. Figure ~\ref{} shows the box plot of run-times for simulation in 2D and 3D.

\begin{figure}
\label{figure1}
\centering
\includegraphics{figure1_v02.png}
\caption{\textbf{Perturbation in DNA.} \textbf{(A)} Chain bending of DNA about random bead $i$ by an angle $\theta$. \textbf{(B)} Crank-shaft rotation of DNA around random bead points $i$, $j$ by an angle $\theta$. The axis of rotation is denoted by the solid red line. The blue and orange balls are the position of beads before and after perturbation respectively. \textit{Image credit - Dr Artem, PC5213 Lecture 6 notes.}}
\end{figure}

\begin{figure}
\label{figure2}
\centering
\includegraphics{figure2_v01.png}
\caption{\textbf{Acceptance probability} for Monte Carlo simulation in \textbf{(A)} 2D, and \textbf{(B)} 3D for different values of $d$, and for a fixed $M = 200$ and $B = 50$ nm. As $d$ increases, the length of DNA increases, but $B$ remains same, which means the longer DNA is more flexible than the shorter DNA. The horizontal black line denotes the acceptance probability = 0.5. Starting with the initial configuration, the DNA is perturbed $10^5$ times by an angle $\delta \in U(-\alpha,\alpha)$. The number of accepted perturbations give the acceptance probability.}
\end{figure}

\section{Results and discussion}
\label{results}
The tangent correlation is efined as 



%\end{multicols}{2}
\end{document}
